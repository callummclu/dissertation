% REMEMBER: You must not plagiarise anything in your report. Be extremely careful.

\documentclass{l4proj}

    
%
% put any additional packages here
%

\begin{document}

%==============================================================================
%% METADATA
\title{Espruino Tools - An asynchronous JavaScript library to control remote embedded devices [JavaScript, Promise, Arduino]}
\author{Callum McLuskey}
\date{Month 00, 2023}

\maketitle

%==============================================================================
%% ABSTRACT
\begin{abstract}
   
\end{abstract}

%==============================================================================

% EDUCATION REUSE CONSENT FORM
% If you consent to your project being shown to future students for educational purposes
% then insert your name and the date below to  sign the education use form that appears in the front of the document. 
% You must explicitly give consent if you wish to do so.
% If you sign, your project may be included in the Hall of Fame if it scores particularly highly.
%
% Please note that you are under no obligation to sign 
% this declaration, but doing so would help future students.
%
%\def\consentname {My Name} % your full name
%\def\consentdate {20 March 2018} % the date you agree
%
\educationalconsent


%==============================================================================
\tableofcontents

%==============================================================================
%% Notes on formatting
%==============================================================================
% The first page, abstract and table of contents are numbered using Roman numerals and are not
% included in the page count. 
%
% From now on pages are numbered
% using Arabic numerals. Therefore, immediately after the first call to \chapter we need the call
% \pagenumbering{arabic} and this should be called once only in the document. 
%
% Do not alter the bibliography style.
%
% The first Chapter should then be on page 1. You are allowed 40 pages for a 40 credit project and 30 pages for a 
% 20 credit report. This includes everything numbered in Arabic numerals (excluding front matter) up
% to but excluding the appendices and bibliography.
%
% You must not alter text size (it is currently 10pt) or alter margins or spacing.
%
%
%==================================================================================================================================
%
% IMPORTANT
% The chapter headings here are **suggestions**. You don't have to follow this model if
% it doesn't fit your project. Every project should have an introduction and conclusion,
% however. 
%
%==================================================================================================================================
\chapter{Introduction}

% reset page numbering. Don't remove this!
\pagenumbering{arabic} 


Why should the reader care about what are you doing and what are you actually doing?
\section{Motivation}

\text Why is this project necessary, preparing compututer science studens for working by developing skills in embedded systems and achieving real life working demos.

\section{Aims /& Goals}

What are the goals of making this project.

\begin{itemize}
    \item
    Simplify development of espruino devices.
    \item
    Bring everything into one place and document it to avoid previous bad documentation.
    \item
    Improve ease of usage/development, autocomplete, easy peer to peer, easy development, easy starting point
    \item
    Provide an easier start for unexperienced developers, orivide an environment that can be initialised in one command.
\end{itemize}



%==================================================================================================================================
\chapter{Background}
What did other people do, and how is it relevant to what you want to do?
\section{Espruino Devices}
\text    
    Espruino are a set of remote embedded devices which run on javascript. Why Espruino, not arduino, micropython (Javascript native)
    
\section{UART}
\text Universal asynchronous receiver-transmitter, this is what the espruino devices are built using. This enables web bluetooth connection between the computer and espruino device.

\section{WebBluetooth}
\text Provides a method for Bluetooth Low Energy(BLE) devices to connect to supported browsers

\section{NPM}
\text a package mangager for javascript using the node.js runtime.

\section{NPX}
\text Node package runner, what is a package runner. This provides a way of creating CLI tools to aid in node development

\section{WebRTC}
\text open source project to allow web browsers to contain real time communication through peer to peer communication. maybe include websocket comparison

\section{Open Source}
\text source code that is freely able to be used and modified. Mention gh organisation.

\section{Existing Projects}
\text Espruino native language / online IDE.

%==================================================================================================================================
\chapter{Analysis/Requirements}
What is the problem that you want to solve, and how did you arrive at it?
\section{Requirements / Problem Specification}
\text Requirements of the project, this could be structured in MOSCOW
\subsection{Functional Requirements}
\text What are functional requirements
\subsection{Non-Functional Requirements}
\text What are non-functional requirements

\section{Specification changes}
\subsection{Peer to Peer}
\text Why does this mater

\subsection{Transpiler}
\text how does this help

\subsection{NPX Tool}
\text how does this help

\subsection{Online Environment}
\text Why?
%==================================================================================================================================
\chapter{Design}
How is this problem to be approached, without reference to specific implementation details? 
\section{Organisations}
\subsection{Github}
\text something about open source, keeping everything together.
\subsection{NPM}
\text scoped packages, keeps everything together.
\section{NPX Tool Repositories}
\text why react, vue, typescript. Why have `--clean-install`, `--peer`
\subsection{Git Submodules}
\text Why.
\subsection{Tags}
\text `--clean-install` `--template` `--peer`
\section{UI/UX Improvements}
\text modals, prototyping (Figma)
\section{Agile Methodologies}
\text choice for software architecture.
%==================================================================================================================================
\chapter{Tools / Technologies}

\section{TypeScript}
\text Improved typing, compiles into js, so on. Why even bother with it why not just use javascript.(static typing and why this makes a difference, improves developments)
\section{NPM}
\text Why not just have a minified js file hosted online.
\section{NPX}
\text What is this why is it important, provides a platform to build a CLI tool that doesn’t need to be downloaded before running a command
\section{UNPKG}
\text Why support both, what does this do. Avoids need to host cdn, is perfectly in sync with npm including package versioning
\section{Azure Devops}
\text Pipelines, issue tracker, explain why over gitlab, travis or jenkins, visualisation of builds for better error handling / management
\section{Webpack}
\text can use new js features and compile into older widely used, enables typescript.Other webpack benefits converts all code into es5 by choice (why es5)
Why not parcel or snowpack or another bundler
\section{React}
\text Why is react used for demos and online IDE vs Vue, Angular, vanilla or any other. Using a framework benefits.
comparing frameworks
\section{JSS}
\text Why style in javascript. No need to package/ bundle css, provides benefits of scss, sass.
\section{peerJS}
\text Why use existing package
\section{qrcode Package}
\text Why generate a qr code with this package
\section{Docusaurus}
\text why this over jekyll, hugo. MDX, agolia search
\section{Husky}
\text what is it (runs git hooks), what are git hooks
auto package versioning
commit sanitizing
%==================================================================================================================================
\chapter{Implementation}
What did you do to implement this idea, and what technical achievements did you make?
\section{Feature / Packages / Web Apps}
You can't talk about everything. Cover the high level first, then cover important, relevant or impressive details.
\subsection{core}
\subsection{peer}
\subsection{uart}
\subsection{create-espruino-app}
\subsection{online IDE}
\subsection{Transpiler}

\section{Documentation}
\text Just why this is so important for a collection of packages.
\section{CI/CD}
\text How the product functionality was checked, Azure using the pipeline to run build checks and tests.

\section{Testing}
\text JEST, javascript testing suite. Cypress,Front end testing suite (this may only be applicable for demo site and online IDE.


\section{Deployment}
\subsection{Packages}
\text How the packages are deployed using azure pipelines ,npm and unpkg to host package.
\subsection{Web Apps}
\text How the web apps are deployed. Vercel, what is vercel and why use it?



\chapter{Evaluation} 
How good is your solution? How well did you solve the general problem, and what evidence do you have to support that?

\section{User Feedback}
\section{Core}
\subsection{Speed Comparison}
\section{Peer}
\subsection{Speed Comparison}
\section{Transpiler}

%==================================================================================================================================
\chapter{Conclusion}    
Summarise the whole project for a lazy reader who didn't read the rest (e.g. a prize-awarding committee).
\section{Summary}
\section{Reflection}
\section{Future Work}
\section{Limitation}

%==================================================================================================================================
%
% 
%==================================================================================================================================
%  APPENDICES  

\begin{appendices}
\chapter{Appendices}
\end{appendices}

%==================================================================================================================================
%   BIBLIOGRAPHY   

% The bibliography style is abbrvnat
% The bibliography always appears last, after the appendices.

\bibliographystyle{abbrvnat}

\bibliography{l4proj}

\end{document}
